%%% The halftitle

\begin{titlepage}
\begin{center}

\textsc{\LARGE Středoškolská odborná činnost}\\[4cm]

\textsc{\Huge Líně, čistě,\\ funkcionálně}

\textit{\large aneb}

\textit{\LARGE Užití funkcionálního programovacího jazyka Haskell k řešení
úloh Ústředního kola ČR Soutěže v programování}\\[2cm]

\textsc{\huge Jan Špaček}\\[10cm]

\textit{\LARGE Ostrava 2013}

\end{center}
\end{titlepage}

% hides the page number
\thispagestyle{empty}
\cleardoublepage

%%% The title

\begin{center}

\textsc{\LARGE Středoškolská odborná činnost}\\[0.6cm]
\textsc{\large Obor SOČ: 18 Informatika}\\[2cm]

\textsc{\Huge Líně, čistě,\\ funkcionálně}

\textit{\large aneb}

\textit{\LARGE Užití funkcionálního programovacího jazyka Haskell k řešení
úloh Ústředního kola ČR Soutěže v programování}\\[2cm]

\textsc{\huge Lazy, pure, functional}

\textit{\Large Using the functional programming language Haskell to solve
tasks from the finals of the Czech Programming Contest}\\[3cm]

\begin{tabularx}{12cm}{
  >{\Large\scshape}p{3cm} 
  >{\LARGE}X 
}
Autor:    & Jan Špaček \\[0.5cm]
% aww, how dirty! :)
Škola:    & Wichterlovo gymnázium \\[0.15cm]
          & Čs.~exilu 669 \\[0.2cm]
          & Ostrava-Poruba \\
\end{tabularx}
\\[3cm]

\textit{\LARGE Ostrava 2013}

\end{center}

\setcounter{page}{1}
\thispagestyle{empty}

\newpage

%%% The declaration

~\\[15cm]
\section*{Prohlášení}

Prohlašuji, že jsem svou práci vypracoval(a) samostatně, použil(a) jsem pouze
podklady (literaturu, SW atd.) uvedené v přiloženém seznamu a postup při
zpracování a dalším nakládání s prací je v souladu se zákonem č. 121/2000 Sb.,
o právu autorském, o právech souvisejících s právem autorským a o změně
některých zákonů (autorský zákon) v platném znění.\\[0.5cm]

V \makebox[3cm]{\dotfill} dne \makebox[4cm]{\dotfill}
podpis: \makebox[6cm]{\dotfill}

\newpage

%%% Thanks

~\\[15cm]
\textit{
Nejen ve své práci jsem využil nepřeberného množství svobodného softwaru.
Zvláště bych rád poděkoval autorům těchto projektů:
GNU/Linux (svobodný operační systém),
Ubuntu a jeho odnož Xubuntu (linuxové distribuce),
Vim (textový editor),
Git (distribuovaný systém správy verzí),
\LaTeX\ (sázecí systém),
Haskell (jazyk, kolem kterého se v této práci všechno točí),
GHC (špičkový překladač Haskellu).
}

\newpage
