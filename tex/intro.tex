\chapter{Úvod}

\section{Co je to Haskell}

Haskell je \textit{de-facto} standardní líný čistě funkcionální jazyk. Jeho
vývoj začal v září roku 1987, přičemž první verze jazyka (Haskell 1.0) byla
vydána 1. března 1990. Postupně následovaly verze 1.1 až 1.4. V roce 1999 byla
vytvořena \uv{stabilní} verze Haskell 98\comment{cite Haskell 98 Report}, která
byla později revidována.\comment{cite Haskell 98 Revised Report}

\marginnote{Citace, citace, citace!}

Od roku 2006 probíhá proces nazývaný Haskell' (Haskell Prime), jehož cílem je
vytvářet nové revize standardu každý rok. První a zatím poslední revizí je
Haskell 2010.\comment{cite Haskell 2010 Report}

\marginnote{úvodní odstavec k vlastnostem}

\subsection{Čistě funkcionální jazyk}

V \emph{imperativních} programovacích jazycích (C, \Cplusplus{}, Java, Ruby,
JavaScript, Befunge, PHP...) jsou algoritmy vyjádřeny jako sekvence kroků --
příkazů, které mění stav programu, zvláště hodnoty proměnných.  Tento přístup
přímo vychází ze strojového kódu počítačů a v dnešní době jde o dominantní
programovací styl.

\emph{Funkcionální} jazyky jsou oproti tomu založeny na vyhodnocování výrazů a
jejich fundamentálním typem je \emph{funkce}. Narozdíl od imperativních jazyků,
kde vyhodnocování výrazů a volání procedur (metod) může mít \emph{vedlejší
účinky}, např. změnu hodnoty proměnných nebo vypsání řetězce na obrazovku,
jediný účinek, který má \uv{zavolání} (aplikace) funkce v čistě funkcionálním
jazyku, je získání jejího výsledku. Jelikož neexistuje žádný stav, na němž by
výsledek funkce mohl záviset, je garantováno, že aplikujeme-li funkci vícekrát
na stejné argumenty, dostaneme vždy stejný výsledek. Tato vlastnost se nazývá
\emph{referenční transparentnost}.

Na funkce je tedy možno nahlížet jako na funkce v matematice, takže pro
programátora i pro kompilátor je snadné zjišťovat a \emph{dokazovat} chování
funkce.

\marginnote{velmi snadná paralelizace!}

\subsection{Líné vyhodnocování}

\emph{Líné vyhodnocování} znamená, že hodnoty výrazů se počítají až ve chvíli,
kdy je to nezbytně nutné. Tím se nejen zvýší efektivita programu (výpočty se
provedou jen když jsou zapotřebí, nemusí se tedy vyhodnotit vůbec), ale hlavně
se programátor zbaví nutnosti zabývat se pořadím vyhodnocování výrazů. Je tedy možné
oddělit kód produkující a kód konzumující data, čímž se zvýší modularita. 

\marginnote{citovat Why FP Matters}

Ve funkcionálních programech je běžné používat nekonečné struktury, např.
nekonečné seznamy či nekonečně se větvící stromy, a nechat vyhodnotit pouze tu
část dat, která je potřeba k získání požadovaného výsledku.

\subsection{Statické typování}

\emph{Statické typování} znamená, že typ každého výrazu je znám již v při
kompilaci programu, takže případné chyby kompilátor odhalí velmi brzy. Není tedy
možné, aby program za běhu zhavaroval na typovou chybu, čímž se eliminuje celá
škála potencionálních bugů -- žádné @t{Segmentation fault}, žádné
@t{NullPointerException}, žádné @t{NoMethodError}. Většinou platí, že když se
kód úspěšně zkompiluje, máme velkou šanci, že bude skutečně fungovat tak, jak si
představujeme.

Narozdíl od jazyků jako Java či C++, jež jsou také staticky typované, je silný
typový systém Haskellu schopen naprostou většinu typů odvodit sám, takže typové
anotace se obvykle používají jen jako druh dokumentace a způsob kontroly určené
hlavně pro člověka (programátora).

\subsection{Typové třídy}

S typovým systémem úzce souvisí \emph{typové třídy}.\footnote{Neplést s třídami
z objektově orientovaných jazyků} Typové třídy byly do Haskellu zavedeny, aby se
vyřešil problém s \uv{přetěžováním} funkcí jako je porovnávání (@t{==}), které
bychom potřebovali použít s větším množstvím typů (operace ekvivalence má smysl
např. pro čísla, řetězce, seznamy, množiny...).

Ukažme si příklad typové třídy @t{Eq}, která slouží k implementaci porovnání
dvou hodnot:

\begin{haskell}
class Eq a where
  (==) :: a -> a -> Bool
\end{haskell}

Tímto deklarujeme, že typ @t{a} patří do třídy @t{Eq} právě tehdy, když
implementuje \emph{metodu} @t{==}.

Mějme dva datové typy:

\begin{haskell}
data Color = Red | Green | Blue
data Bit = On | Off
\end{haskell}

Pro oba tyto typy určitě dává operace porovnání smysl, proto můžeme nadefinovat
\emph{instanci} třídy @t{Eq}:

\begin{haskell}
instance Eq Color where
  Red   == Red   = True
  Green == Green = True
  Blue  == Blue  = True
  _     == _     = False

instance Eq Bit where
  On  == On  = True
  Off == Off = True
  _   == _   = False
\end{haskell}

Touto deklarací specifikujeme, že typy @t{Color} a @t{Bit} jsou instancemi třídy @t{Eq}, a
poskytneme implementaci metody @t{==}. Nyní můžeme používat funkci @t{==} s
barvami i bity, např. @t{Red == Red} vrátí @t{True} a @t{On == Off} vrátí
@t{False}.

Všimněte si, že nemůžeme porovnávat barvy a bity -- napíšeme-li @t{Red == On},
kompilátor ohlásí typovou chybu, jelikož funkce @t{==} akceptuje pouze hodnoty
stejného typu.

Kdybychom chtěli funkci, která otestuje, jestli se daný prvek nachází v
seznamu, mohli bychom ji nadefinovat takto:

\begin{haskell}
elem :: Eq a => a -> [a] -> Bool
elem _ [] = False
elem x (y:ys) = if x == y then True else elem x ys
\end{haskell}

Typ této funkce, @t{Eq a => a -> [a] -> Bool}, odráží skutečnost, že tato funkce
je definovaná pouze pro takové typy @t{a}, které náleží do třídy @t{Eq}. Můžeme
ji tedy použít jak na seznam barev (@t{[Color]}), tak na seznam bitů
(@t{[Bit]}).

Instancemi tříd samozřejmě nemusí být jen takovéto jednoduché typy.  Můžeme si
nadefinovat typ reprezentující binární strom:

\begin{haskell}
data Tree a = Node (Tree a) (Tree a) | Leaf a
\end{haskell}

Tento typ bychom obratem mohli učinit instancí třídy @t{Eq}:

\begin{haskell}
instance Eq a => Eq (Tree a) where
  Leaf x == Leaf y         = x == y
  Node l1 r1 == Node l2 r2 = l1 == l2 && r1 == r2
  _ == _ = False
\end{haskell}

Tato instance deklaruje, že pro všechny typy @t{a} je typ @t{Tree a} instancí
třídy @t{Eq}, platí-li, že typ @t{a} je rovněž instancí @t{Eq}.

\subsection{Monadický vstup/výstup}

Jak už bylo zmíněno výše, vyhodnocování výrazů v čistě funkcionálním jazyce
nemůže mít žádné vedlejší efekt a musí být referenčně transparentní. Co když ale
potřebujeme vykonat nějakou vstupně/výstupní operaci, např. přečíst znak, který
uživatel napsal na klávesnici, nebo zapsat soubor na disk? Takové \uv{funkce} by
ale určitě vedlejší efekt měly a referenčně transparentní jistě také nejsou.

Většina jiných funkcionálních jazyků tento problém řeší tak, že obětuje
\emph{čistotu} a takové \uv{nefunkcionální funkce} povoluje, přičemž starosti s
porušením referenční transparentnosti nechává na programátorovi. Tento přístup
ovšem není možný v líně vyhodnocovaném jazyku, jelikož nemáme žádnou kontrolu
nad tím, v jakém pořadí a jestli vůbec se funkce \uv{zavolají}.

\subsubsection{\texorpdfstring{Typ @t{IO}}{Typ IO}}

Haskell proto používá jinou techniku, která umožňuje zachovat funkcionální
čistotu i líné vyhodnocování. Zavádí typ @t{IO a}, který reprezentuje
\emph{vstupně/výstupní operaci}, jejíž výsledek je typu @t{a}. Dostaneme tedy
takovéto typy:

\begin{haskell}
getChar :: IO Char
getLine :: IO String
putLine :: String -> IO ()

readFile :: FilePath -> IO String
writeFile :: FilePath -> String -> IO ()
\end{haskell}

@t{getChar} je vstupně/výstupní operace, která přečte jeden znak zadaný
uživatelem a jejím výsledkem je tento znak, tedy typ @t{Char}. Obdobně
@t{getLine} přečte celý řádek a její výsledek je typu @t{String}. Chceme-li
naopak řádek vypsat, můžeme použít @t{putLine}.\footnote{Ve standardní knihovně
se tato funkce jmenuje @t{putStrLn}} @t{putLine} je funkce, které
předáme řetězec, jež si přejeme vypsat, a dostaneme vstupně/výstupní operaci,
která tento řetězec vypíše. Jelikož tato operace nemá žádný smyslupný výsledek,
vrací tzv. \emph{nulový typ} @t{()}, který má jedinou hodnotu (rovněž se zapisuje
@t{()}) a používá se jako jistá \uv{výplň}.

Abychom přečetli soubor, musíme znát jeho umístění, proto je @t{readFile}
funkce, které předáme cestu k souboru\footnote{@t{FilePath} je synonym k typu
  @t{String}, který se ve standardní knihovně používá pro označení cest k
souboru.} a dostaneme vstupně/výstupní operaci, jejíž výsledek bude obsah
souboru jako řetězec (@t{String}). Chceme-li zapsat data do souboru, použijeme
funkci @t{writeFile}, které předáme dva argumenty -- cestu k souboru a řetězec
znaků, které si přejeme zapsat -- a dostaneme IO operaci. Výsledek této operace
je opět @t{()}.

Existuje pouze jedna možnost, jak vykonat operaci reprezentovanou typem @t{IO}
-- definovat ji jako proměnnou\footnote{\uv{Proměnná} samozřejmě neznamená, že
  se její hodnota nějakým způsobem mění; tento pojem se používá podobně jako v
matematice. V imperativním jazyku bychom řekli \uv{konstanta}.} @t{main} v
modulu @t{Main}. Tato hodnota má typ @t{IO $\alpha$}. Spustit program v Haskellu
tedy vlastně znamená vyhodnotit operaci, která je přiřazena do @t{main}, a
výsledek typu @t{$\alpha$} zahodit.

Pokud bychom tedy po programu chtěli, ať spočte jednoduchý příklad a výsledek
vypíše, mohli bychom nadefinovat @t{main} takto (funkce @t{++} slouží ke spojení
dvou řetězců a funkce @t{show} převede číslo na řetězec):

\begin{haskell}
main = putStrLn ("Jedna plus jedna je " ++ show (1+1))
\end{haskell}

\subsubsection{\texorpdfstring
  {Spojování operací pomocí @t{>>=} a @t{>>}}
  {Spojování operací pomocí >>= a >>}}

Co kdybychom ale chtěli provést více operací, například se zeptat uživatele na
jméno a pak ho pozdravit? Potřebujeme nějakým způsobem \uv{slepit} dvě IO
operace, a to takovým způsobem, aby druhá mohla využít výsledek první. K tomu
slouží funkce @t{>>=}, někdy též nazývaná \uv{bind}. Její typ je takovýto:

\begin{haskell}
(>>=) :: IO a -> (a -> IO b) -> IO b
\end{haskell}

Jako první argument ji předáme první vstupně/výstupní operaci, kterou si přejeme
vykonat a její výsledek dát jako vstupní hodnotu funkci předané jako druhý
argument.

S její pomocí můžeme uživatele pozdravit takto: \footnote{Syntaxe
@t{\textbackslash x -> E} v Haskellu znamená anonymní (lambda) funkci s
parametrem @t{x} a tělem @t{E}.}

\begin{haskell}
main = getStrLn >>= (\name -> putStrLn ("Ahoj " ++ name))
\end{haskell}
  
Jak se ale uživatel doví, že po něm chceme, aby napsal svoje jméno? Nejprve mu
musíme sdělit, že po něm požadujeme zadat jméno, a až poté jej přečíst a
pozdravit. Opět můžeme využít @t{putStrLn} a @t{>>=}:

\begin{haskell}
main = 
  putStrLn "Kdo tam?" >>= (\_ ->
    getStrLn >>= (\name ->
      putStrLn ("Ahoj " ++ name)))
\end{haskell}

Výsledek z prvního @t{putStrLn} nás nazajímá (koneckonců to je pouze nulový typ
@t{()}), takže jsme jej přiřadili do speciální \uv{proměnné} @t{_}, která
funguje jako \uv{černá díra} -- můžeme do ní přiřazovat nepotřebné údaje. Tento
způsob zacházení s výsledky vstupně/výstupních operací je poměrně častý, proto
je v Haskellu definována funkce @t{>>} s typem @t{IO a -> IO b -> IO b}, která
je podobná @t{>>=}, s tím rozdílem, že první operaci sice provede, ale její
výsledek zahodí a následně provede druhou operaci (nepředáváme jí tedy funkci),
jejíž výsledek vrátí.

\begin{haskell}
main = 
  putStrLn "Kdo tam?" >>
    getStrLn >>= (\name ->
      putStrLn ("Ahoj " ++ name))
\end{haskell}

Tento kód ale není příliš přehledný a pokud bychom jej chtěli ještě dále rozšířit, mohli
bychom se v něm brzy ztratit. Protože používání @t{>>=} a @t{>>} je v Haskellu
velmi časté, obsahuje jazyk tzv. @t{do}-syntaxi, která nám umožňuje zapisovat
série operací o něco úhledněji:

\begin{haskell}
main = do
  putStrLn "Kdo tam?"
  name <- getStrLn
  putStrLn ("Ahoj " ++ name)
\end{haskell}
