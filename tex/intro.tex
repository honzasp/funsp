\chapter{Úvod}

Funkcionální programování má původ v $\lambda$-kalkulu, matematickém formálním
systému, jež popisuje výpočty jako vyhodnocování výrazů tvořených funkcemi.  S
funkcemi se v $\lambda$-kalkulu pracuje jako s hodnotami\footnote{Základní
$\lambda$-kalkulus dokonce žádné jiné hodnoty než funkce nezná.} a lze je
\emph{aplikovat} na argument a získat jejich výsledek nebo je vytvořit pomocí
\emph{$\lambda$-abstrakce}. Tento systém je svou výpočetní silou ekvivalentní
Turingovu stroji, je v něm tedy možno vyjádřit jakýkoli strojem proveditelný
výpočet.

V \emph{imperativních} programovacích jazycích (C, \Cplusplus{}, Java, Ruby,
JavaScript, Befunge, PHP...) jsou algoritmy vyjádřeny jako sekvence kroků --
příkazů, které mění stav programu, zvláště hodnoty proměnných. Postupným
vykonáváním těchto kroků se provádí výpočet. Tento přístup přímo vychází ze
strojového kódu počítačů a v dnešní době jde o dominantní programovací styl.

Ve funkcionálních jazycích se naproti tomu výpočty provádí \emph{vyhodnocováním
výrazů}. Aplikace (\uv{zavolání}) funkce má jediný účinek, a to je získání
jejího výsledku. Jelikož neexistuje žádný stav, na němž by výsledek funkce mohl
záviset, je garantováno, že aplikujeme-li funkci vícekrát na stejné argumenty,
dostaneme vždy stejný výsledek. Tato vlastnost se nazývá \emph{referenční
transparentnost}.

Na funkce je tedy možno nahlížet podobně jako na funkce v matematice, takže pro
programátora i pro kompilátor je snadné zjišťovat a \emph{dokazovat} chování
funkcí. Kompilátor může kód Haskellu snadno optimalizovat
\cite{santos1995compilation}, takže programátor může v programech používat
vysokoúrovňové konstrukce bez obav z negativního dopadu na rychlost programu.

Další vlastností funkcionálních jazyků je snadná paralelizace. Potřebujeme-li
zjistit hodnotu dvou výrazů, můžeme bez obav každý výpočet provést na
samostatném procesoru, jelikož máme zaručeno, že se výpočty nebudou navzájem
ovlivňovat.

\section{Haskell}

Haskell je \textit{de-facto} standardní líný čistě funkcionální jazyk. Jeho
vývoj začal v září roku 1987, přičemž první verze jazyka (Haskell 1.0) byla
vydána 1. března 1990. Postupně následovaly verze 1.1 až 1.4. V roce 1999 byla
vytvořena \uv{stabilní} verze Haskell 98, která byla později revidována
\cite{jones2003haskell}. \cite{hudak2007history}

Od roku 2006 probíhá proces nazývaný Haskell' (Haskell Prime), jehož cílem je
vytvářet nové revize standardu každý rok. První a zatím poslední revizí je
Haskell 2010 \cite{haskellreport2010}.

V následujících podsekcích budou představeny základní vlastnosti jazyka Haskell,
v práci ovšem budeme předpokládat jistou základní znalost funkcionálního
programování v tomto jazyku. Pokud se čtenář s tímto stylem programování ještě
nesetkal, autor mu může doporučit knihu slovinského studena Mirana Lipovači
\emph{Learn You a Haskell for Great Good} \cite{lipovaca2011learn}, která je
volně dostupná na Internetu jak v původní anglické
verzi,\footnote{\url{http://learnyouahaskell.com/}} tak v částečném českém
překladu.\footnote{\url{http://naucte-se.haskell.cz/}} Práce by nicméně měla být
srozumitelná i se základními znalostmi \uv{klasického} programování.

\subsection{Čistý funkcionální jazyk}

Haskell je \emph{čistý} funkcionální jazyk,\footnote{Někdy se používá rovněž
označení \uv{čistě funkcionální jazyk}, které ovšem není úplně přesné.} což
znamená, že \emph{všechny} funkce jsou referenčně transparentní a nemají žádný
vnitřní stav, jsou tedy prosty vedlejších účinků.  

Většina jazyků, které bývají označováný jako funkcionální, totiž funkce s
vedlejšími účinky povoluje, například pro vykonávání vstupně-výstupních operací
nebo používání měnitelných datových struktur, potřebných pro efektivní
implementaci některých algoritmů. Haskell ovšem pro tyto účely používá jiné
prostředky, především \emph{monády}, které si představíme v podsekci
\ref{subsec:intro-monad} a které umožňují zachovat čistotu jazyka.

\subsection{Líné vyhodnocování}

Haskell je \emph{líně vyhodnocovaný} jazyk, což znamená, že hodnoty výrazů se
vyhodnocují až ve chvíli, kdy je to nezbytně nutné, přičemž hodnoty, které
nejsou potřeba, se nevyhodnotí vůbec. Tím se nejen zvýší efektivita programu
(neprovádí se zbytečné výpočty), ale hlavně se programátor zbaví nutnosti
zabývat se pořadím vyhodnocování výrazů. Je tedy možné například snadno oddělit
kód produkující a kód konzumující data, čímž se zvýší modularita.
\cite{hughes1989functional}

Ve funkcionálních programech je běžné používat nekonečné struktury, např.
nekonečné seznamy či nekonečně se větvící stromy, a nechat vyhodnotit pouze tu
část dat, která je potřeba k získání požadovaného výsledku.

\subsection{Statické typování}

\emph{Statické typování} znamená, že typ každého výrazu je znám již v při
kompilaci programu, takže případné chyby kompilátor odhalí velmi brzy. Není tedy
možné, aby program za běhu zhavaroval na typovou chybu, čímž se eliminuje celá
škála potencionálních bugů. Většinou platí, že když se kód úspěšně zkompiluje,
máme velkou šanci, že bude už napoprvé skutečně fungovat tak, jak si
představujeme.

Narozdíl od jazyků jako Java či \Cplusplus, jež jsou také staticky typované, je
mnohem silnější typový systém Haskellu schopen naprostou většinu typů odvodit
sám, takže typové anotace se obvykle používají jen jako druh dokumentace a
způsob kontroly určené pro člověka (programátora).

\subsection{Typové třídy}

S typovým systémem úzce souvisí \emph{typové třídy}. Typové třídy byly do
Haskellu zavedeny, aby se vyřešil problém s \uv{přetěžováním} funkcí jako je
například porovnávání (@t{==}), které bychom potřebovali použít s větším
množstvím typů (operace ekvivalence má smysl např. pro čísla, řetězce, seznamy,
množiny...).

Ukažme si příklad typové třídy @t{Eq}, která slouží k implementaci porovnání
dvou hodnot:

\begin{haskell}
class Eq a where
  (==) :: a -> a -> Bool
\end{haskell}

Tímto deklarujeme, že typ @t{a} patří do třídy @t{Eq} právě tehdy, když
implementuje \emph{metodu} @t{==}.

Mějme dva datové typy:

\begin{haskell}
data Color = Red | Green | Blue
data Bit = On | Off
\end{haskell}

Pro oba tyto typy určitě dává operace porovnání smysl, proto můžeme nadefinovat
\emph{instanci} třídy @t{Eq}:\footnote{Haskell umožňuje pro datové typy
  automaticky odvodit instance tříd ze standardní knihovny, jako je @t{Eq} nebo
@t{Ord}, pomocí klauzule @t{deriving}, takže jen zřídka definujeme operaci
@t{==} takto \uv{ručně}.}

\begin{haskell}
instance Eq Color where
  Red   == Red   = True
  Green == Green = True
  Blue  == Blue  = True
  _     == _     = False

instance Eq Bit where
  On  == On  = True
  Off == Off = True
  _   == _   = False
\end{haskell}

Touto deklarací specifikujeme, že typy @t{Color} a @t{Bit} jsou instancemi
třídy @t{Eq}, a poskytneme implementaci metody @t{==}. Nyní můžeme používat
funkci @t{==} s barvami i bity, např. @t{Red == Red} vrátí @t{True} a @t{On ==
Off} vrátí @t{False}. Zároveň ale nemůžeme omylem porovnávat barvy a bity --
napíšeme-li @t{Red == On}, kompilátor ohlásí typovou chybu, jelikož funkce
@t{==} akceptuje pouze hodnoty stejného typu.

Kdybychom chtěli funkci, která otestuje, jestli se daný prvek nachází v
seznamu, mohli bychom ji nadefinovat takto:

\begin{haskell}
elem :: Eq a => a -> [a] -> Bool
elem _ [] = False
elem x (y:ys) = if x == y then True else elem x ys
\end{haskell}

Typ této funkce, @t{Eq a => a -> [a] -> Bool}, odráží skutečnost, že tato funkce
je definovaná pouze pro takové typy @t{a}, které náleží do třídy @t{Eq}. Můžeme
ji tedy použít jak na seznam barev (@t{[Color]}), tak na seznam bitů
(@t{[Bit]}).

Instancemi tříd samozřejmě nemusí být jen takovéto jednoduché typy.  Můžeme si
nadefinovat typ reprezentující binární strom:

\begin{haskell}
data Tree a = Node (Tree a) (Tree a) | Leaf a
\end{haskell}

Tento typ bychom obratem mohli učinit instancí třídy @t{Eq}:

\begin{haskell}
instance Eq a => Eq (Tree a) where
  Leaf x == Leaf y         = x == y
  Node l1 r1 == Node l2 r2 = l1 == l2 && r1 == r2
  _ == _ = False
\end{haskell}

Tato instance deklaruje, že pro všechny typy @t{a} náleží typ @t{Tree a} do
třídy @t{Eq}, platí-li, že typ @t{a} náleží do třídy @t{Eq}.

\subsection{Monadický vstup/výstup}
\label{subsec:intro-monad}

Jak už bylo zmíněno výše, vyhodnocování výrazů v čistě funkcionálním jazyce
nemůže mít žádné vedlejší efekty a musí být referenčně transparentní. Co když
ale potřebujeme vykonat nějakou vstupně/výstupní operaci, např. přečíst znak,
který uživatel napsal na klávesnici, nebo zapsat soubor na disk? Takové
\uv{funkce} by určitě vedlejší efekt měly a referenčně transparentní jistě také
nejsou.

Většina jiných funkcionálních jazyků tento problém řeší tak, že obětuje
\emph{čistotu} a takové \uv{nefunkcionální funkce} povoluje, přičemž starosti s
porušením referenční transparentnosti nechává na programátorovi. Tento přístup
ovšem není možný v líně vyhodnocovaném jazyku, jelikož nemáme žádnou kontrolu
nad tím, v jakém pořadí a jestli vůbec se funkce \uv{zavolají}.

\subsubsection{\texorpdfstring{Typ @t{IO}}{Typ IO}}

Haskell proto používá jinou techniku, která umožňuje zachovat funkcionální
čistotu i líné vyhodnocování. Standardní knihovna poskytuje typ @t{IO a}, který
reprezentuje \emph{vstupně/výstupní operaci}, jejíž výsledek je typu @t{a}.

Ukažme si příklad několika operací ze standardní knihovny:

\begin{haskell}
getChar :: IO Char
getLine :: IO String
putStrLn :: String -> IO ()

readFile :: FilePath -> IO String
writeFile :: FilePath -> String -> IO ()
\end{haskell}

@t{getChar} je vstupně/výstupní operace, která přečte jeden znak zadaný
uživatelem a jejím výsledkem je tento znak, tedy typ @t{Char}. Obdobně
@t{getLine} přečte celý řádek a její výsledek je typu @t{String}. Chceme-li
naopak řádek vypsat, můžeme použít @t{putStrLn}, což je funkce, které předáme
řetězec, jež si přejeme vypsat, a dostaneme vstupně/výstupní operaci, která
tento řetězec vypíše. Jelikož tato operace nemá žádný smyslupný výsledek, vrací
tzv. \emph{nulový typ} @t{()}, který má jedinou hodnotu (rovněž se zapisuje
@t{()}) a používá se jako \uv{výplň}.\footnote{Jde o jistou obdobu typu
  @t{void} z jazyka C nebo @t{nil} či @t{null} z dynamických jazyků jako Ruby či
JavaScript, která je ovšem typově bezpečná.}

Abychom přečetli soubor, musíme znát jeho umístění, proto je @t{readFile}
funkce, které předáme cestu k souboru\footnote{Typ @t{FilePath} je synonym k
  typu @t{String}, který se ve standardní knihovně používá pro označení cest k
souboru.} a dostaneme vstupně/výstupní operaci, jejíž výsledek bude obsah
souboru jako řetězec (@t{String}). Chceme-li zapsat data do souboru, použijeme
funkci @t{writeFile}, které předáme dva argumenty -- cestu k souboru a řetězec
znaků, které si přejeme zapsat -- a dostaneme IO operaci. Výsledek této operace
je opět @t{()}.

Existuje pouze jedna možnost, jak vykonat operaci reprezentovanou typem @t{IO}
-- definovat ji jako proměnnou @t{main} v modulu
@t{Main}.\footnote{\uv{Proměnná} samozřejmě neznamená, že se její hodnota
  nějakým způsobem mění; tento pojem se používá podobně jako v matematice. V
imperativním jazyku bychom řekli \uv{konstanta}.} Tato hodnota má typ @t{IO
$\alpha$}. Spustit program v Haskellu tedy vlastně znamená vyhodnotit operaci,
která je přiřazena do @t{main}, a výsledek typu @t{$\alpha$} zahodit.

Pokud bychom tedy po programu chtěli, ať spočte jednoduchý příklad a výsledek
vypíše, mohli bychom nadefinovat @t{main} takto (funkce @t{++} slouží ke spojení
dvou řetězců a funkce @t{show} převede číslo na řetězec):

\begin{haskell}
main = putStrLn ("Jedna plus jedna je " ++ show (1+1))
\end{haskell}

\subsubsection{\texorpdfstring
  {Spojování operací pomocí @t{>>=} a @t{>>}}
  {Spojování operací pomocí >>= a >>}}

Co kdybychom ale chtěli provést více operací, například se zeptat uživatele na
jméno a pak ho pozdravit? Potřebujeme nějakým způsobem \uv{slepit} dvě IO
operace, a to takovým způsobem, aby druhá mohla využít výsledek první. K tomu
slouží funkce @t{>>=}, někdy též nazývaná \uv{bind}. Její typ je takovýto:

\begin{haskell}
(>>=) :: IO a -> (a -> IO b) -> IO b
\end{haskell}

Jako první argument jí předáme první vstupně/výstupní operaci, kterou si přejeme
vykonat a její výsledek dát jako vstupní hodnotu funkci předané jako druhý
argument. Výsledek této funkce \uv{zabalený} v typu @t{IO} je výsledkem celé
@t{>>=}.

S její pomocí můžeme uživatele pozdravit takto: \footnote{Syntaxe
@t{\textbackslash x -> E} znamená anonymní (lambda) funkci s parametrem @t{x} a
tělem @t{E}.}

\begin{haskell}
main = getStrLn >>= (\name -> putStrLn ("Ahoj " ++ name))
\end{haskell}
  
Jak se ale uživatel doví, že po něm chceme, aby napsal svoje jméno? Nejprve mu
musíme sdělit, že po něm požadujeme zadat jméno, a až poté jej přečíst a
pozdravit. Opět můžeme využít @t{putStrLn} a @t{>>=}:

\begin{haskell}
main = 
  putStrLn "Kdo tam?" >>= (\_ ->
    getStrLn >>= (\name ->
      putStrLn ("Ahoj " ++ name)))
\end{haskell}

Výsledek z prvního @t{putStrLn} nás nazajímá (koneckonců to je pouze nulový typ
@t{()}), takže jsme jej přiřadili do speciální \uv{proměnné} @t{_}, která
funguje jako \uv{černá díra} -- můžeme do ní přiřazovat nepotřebné údaje. Tento
způsob zacházení s výsledky vstupně/výstupních operací je poměrně častý, proto
je definována funkce @t{>>} s typem @t{IO a -> IO b -> IO b}, která je podobná
@t{>>=}, s tím rozdílem, že první operaci sice provede, ale její výsledek zahodí
a následně provede druhou operaci (nepředáváme jí tedy funkci), jejíž výsledek
vrátí.

\begin{haskell}
main = 
  putStrLn "Kdo tam?" >>
    getStrLn >>= (\name ->
      putStrLn ("Ahoj " ++ name))
\end{haskell}

Tento kód ale není příliš přehledný a pokud bychom jej chtěli ještě dále
rozšířit, mohli bychom se v něm brzy ztratit. Protože zřetězené používání
@t{>>=} a @t{>>} je v Haskellu velmi časté, obsahuje jazyk tzv. @t{do}-syntaxi,
která nám umožňuje zapisovat série takovýchto operací o něco úhledněji:

\begin{haskell}
main = do
  putStrLn "Kdo tam?"
  name <- getStrLn
  putStrLn ("Ahoj " ++ name)
\end{haskell}

\subsubsection{Monády}

Typ @t{IO} je jen jedním z příkladů \emph{monád}. Monády, původně koncept z
teorie kategorií, je v Haskellu možno považovat za \emph{výpočty} nebo
\emph{akce}, které je možno \emph{skládat}, tedy přesměrovat \emph{výstup} z
jedné monády do druhé.

Do Haskellu jsou monády zahrnuty pomocí typové třídy @t{Monad} ze standardní
knihovny jazyka. Tato třída je definována takto:

\begin{haskell}
class Monad m where
  (>>=)  :: m a -> (a -> m b) -> m b
  (>>)   :: m a -> m b -> m b
  m >> k  = m >>= \_ -> k

  return :: a -> m a
\end{haskell}

Funkce @t{>>=} (nazývaná \uv{bind}) a @t{>>}, které jsme si výše ukázali při
použití s typem @t{IO}, jsou tedy ve skutečnosti definované pro všechny monády.
Funkce @t{return} pak pro libovolnou hodnotu vytvoří monádu, jejímž výstupem je
tato hodnota.

Kromě typu @t{IO} jsou ze standardní knihovny monádami například i typy @t{[]}
(seznam), @t{Maybe}, @t{Either e} či @t{(->) a} (funkce, jejímž argumentem je
typ @t{a}).  V podsekci \ref{subsec:banshee-st} na straně
\pageref{subsec:banshee-st} si představíme monádu @t{ST s}, kterou je možno
považovat za \uv{odlehčený} typ @t{IO}, umožňující v čistém kódu využívat
měnitelné datové struktury a efektivně implementovat algoritmy, které takovéto
struktury vyžadují.

\section{Dokumentace}

Veškeré knihovny použité v této práci se, včetně dokumentace, nachází na serveru
Hackage.\footnote{\url{http://hackage.haskell.org/packages/hackage.html}} Pro
vyhledávání v dokumentaci je možno použít vyhledávač
Hoogle.\footnote{\url{http://www.haskell.org/hoogle/}} Tento vyhledávač umožňuje
vyhledávat nejen podle jména požadované funkce či třídy, ale i podle přibližného
\emph{typu}, což z něj činí velmi užitečný nástroj.

\section{Řešené úlohy}

Řešené úlohy pochází z Ústředních kol Soutěže v programování, vyhlašované
Ministerstvem školství, mládeže a tělovýchovy a pořádané Národním institutem
dětí a mládeže, z let 2010 (soutěžní úloha s Krunimírem) a 2012 (úloha s Bílou
paní),

Tyto úlohy byly vybrány, protože nejsou ani příliš rozsáhlé, ani příliš
jednoduché, a není je možno podezírat, že by byly vytvořeny funkcionálnímu jazyku
\uv{na míru}.

Zadání obou původních úloh je možno nalézt jak na Internetu
(\cite{krunimir-task}, \cite{banshee-task}), tak přetištěné v této práci na
stranách \pageref{pdf:krunimir} a \pageref{pdf:banshee}.

\section{Technická stránka práce}

V této práci jsou popsány dva programy, jež řeší tyto úlohy. Kód je zapsán ve
stylu \emph{literárního programování}, kdy se volně střídá kód srozumitelný pro
počítač a komentář pro člověka, což znamená, že zdrojový kód Haskellu obou
programů je zároveň zdrojovým kódem systému \LaTeX, kterým je tato práce
vysázena.

Texty kapitol \ref{chap:krunimir} a \ref{chap:banshee} jsou tedy kompletní
programy, které je možno přeložit a spustit. Tímto je zajištěno, že zde
prezentovaný kód je plně funkční.

Zdrojový kód práce je spravován pomocí distribuovaného systému správy verzí Git.
Repozitář je volně přístupný na serveru
\href{http://github.com}{github.com},\footnote{\url{http://github.com}} adresa
práce je \url{https://github.com/honzasp/funsp}. Detailnější popis jednotlivých
souborů a složek se nachází v souboru @t{README}.

Jako kompilátor byl využit GHC (Glasgow Haskell Compiler), který je dnes mezi
překladači Haskellu dominantní.
