\chapter{Úvod}

Naprostá většina z dnešních široce využívaných jazyků (C, C++, C\#, Java,
Pascal, JavaScript, Ruby, Python, PHP...) je \emph{imperativní}, což znamená, že
algoritmy jsou v nich vyjádřeny jako posloupnost příkazů, které mění stav
programu.

Alternativní přístup nabízí jazyky \emph{funkcionální}. Výpočty se v nich
provádí vyhodnocováním výrazů, základní operací je aplikace (volání) funkce na
její argumenty. Základní rozdíl oproti imperativním jazykům je v tom, že
výsledek funkce závisí pouze na předaných argumentech. Taková funkce nemůže mít
žádný jiný efekt než vrácení svého výsledku. Tato vlastnost se nazývá
\emph{referenční transparentnost}.

Pro funkcionální programování je typické, že nespecifikujeme \emph{jak} se
dostaneme k výsledky, ale \emph{co} je výsledek. Programy jsou kratší a často i
přehlednější, u funkcí je snadné dokazovat jejich korektnost.

\marginnote{Co je Haskell? Proč Haskell?}

Další vlastností Haskellu je jeho \emph{líné vyhodnocování}. To znamená, že
hodnota výrazu je zjišťována až když je to nezbytně nutné. Tímto se nejen
zbavíme zbytečných výpočtů, jejichž výsledky nejsou zapotřebí, ale také můžeme
snadno pracovat i s nekonečnými strukturami.

\marginnote{Co řešíme? Proč zrovna tyto úlohy?}
