\begin{adjustwidth}{1.5cm}{1.5cm}

% there is a copy of this abstract in README which must be kept in sync!

\section*{Abstrakt}

Funkcionální programování, vycházející z $\lambda$-kalkulu, tvoří alternativu k
dnes široce využívanému imperativnímu přístupu k programování, reprezentovanému
jazyky jako C, \Cplusplus{}, Java, \Csh{}, Objective-C, JavaScript, Python, Ruby
či PHP.

V této práci je představeno řešení dvou úloh Ústředního kola ČR Soutěže v
programování z let 2010 a 2012 s využitím líného funkcionálního jazyka Haskell.
První úlohou je implementovat jednoduchý programovací jazyk, podobný jazyku
Logo, sloužící k vykreslování želví grafiky. Druhá úloha se zabývá hledáním
nejkratší cesty v bludišti s pohyblivými překážkami a omezenou schopností
procházet zdmi.

Cílem práce je prezentovat funkcionální programování a techniky, které se v něm
využívají, a ukázat, že tento programovací styl není pouze akademickou
kuriozitou, ale že se dá využít i k řešení \uv{skutečných} úloh.

\paragraph*{Klíčová slova}
Funkcionální programování; Haskell; želví grafika; programovací jazyky; hledání
cest.

\section*{Abstract}

\paragraph*{Keywords}
Functional programming; Haskell; turtle graphics; programming languages;
pathfinding.

\end{adjustwidth}
