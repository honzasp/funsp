\begin{adjustwidth}{1.5cm}{1.5cm}

% there is a copy of this abstract in README which must be kept in sync!

\section*{Abstrakt}

Funkcionální programování, vycházející z $\lambda$-kalkulu, tvoří alternativu k
dnes široce využívanému imperativnímu přístupu k programování, reprezentovanému
jazyky jako C, \Cplusplus{}, Java, \Csh{}, Objective-C, JavaScript, Python, Ruby
či PHP.

V této práci je představeno řešení dvou úloh Ústředního kola ČR Soutěže v
programování z let 2010 a 2012 s využitím líně vyhodnocovaného funkcionálního
jazyka Haskell. První úloha je implementace jednoduchého programovacího jazyka,
podobného jazyku Logo, sloužícího k vykreslování želví grafiky. Druhá úloha se
zabývá hledáním nejkratší cesty v bludišti s pohyblivými překážkami a omezenou
schopností procházet zdmi.

Cílem práce je prezentovat funkcionální programování a techniky, které se v něm
využívají, a ukázat, že tento programovací styl není pouze předmětem
akademického výzkumu, ale že se dá využít i k řešení \uv{skutečných} úloh.

\paragraph*{Klíčová slova}
Funkcionální programování; Haskell; želví grafika; programovací jazyky; hledání
cest.

\section*{Abstract}

Functional programming has its roots in $\lambda$-calculus and is an alternative
to the mainstream imperative paradigm, represented by languages as C,
\Cplusplus{}, Java, \Csh{}, Objective-C, JavaScript, Python, Ruby or PHP.

In this paper we present solutions to two tasks from the Czech Programming
Contest, a competition for secondary and high school students, from the final
rounds held in 2010 and 2012. The first task is an implementation of a simple
programming language drawing turtle graphics based on Logo. The second task is a
variation of pathfinding in a maze with moving obstacles and a limited ability
to get through the walls.

The goal of the paper is to present functional programming and its technique and
to show that this style of programming is not just a subject of academic
research, but might be used to solve ``real-world'' problems.

\paragraph*{Keywords}
Functional programming; Haskell; turtle graphics; programming languages;
pathfinding.

\end{adjustwidth}
