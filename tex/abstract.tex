\begin{adjustwidth}{1.5cm}{1.5cm}

\section*{Abstrakt}

Funkcionální programování, vycházející z $\lambda$-kalkulu, tvoří alternativu k
dnes široce využívanému imperativnímu přístupu k programování, reprezentovanému
jazyky jako C, \Cplusplus{}, Java, \Csh{}, Objective-C, JavaScript, Python, Ruby
či PHP.

Tyto jazyky jsou založeny na postupném vykonávání příkazů, jež mění stav
programu (zvláště hodnoty proměnných). Naproti tomu funkcionální jazyky provádí
výpočty vyhodnocováním výrazů. Základní stavební prvky -- funkce -- jsou od
imperativních protějšků (ať už procedur, metod nebo \uv{funkcí} jako v jazyce C)
odlišné tím, že jejich výsledek vždy závisí pouze na vstupních argumentech, čímž
se přibližují matematickým funkcím.

Algoritmy zapsané ve funkcionálním jazyce tudíž nemají podobu série kroků, jež
se musí postupně vykonat a jež určují \emph{jak} se k výsledku dostat, ale spíše
popisují \emph{co} je jejich výsledkem.

V této práci je představeno řešení dvou úloh Ústředního kola ČR Soutěže v
programování z let 2010 a 2012 užitím funkcionálního programovacího jazyka
Haskell. Cílem je čtenáře seznámit s vysokoúrovňovými koncepty, které se ve
funkcionálních programech používají, a jejich aplikací při řešení
\uv{skutečných} úloh.

\paragraph*{Klíčová slova}
Funkcionální programování; Haskell; želví grafika; AI.

\section*{Abstract}

Functional programming, rooted in $\lambda$-calculus, is an alternative to the
mainstream imperative approach to programming, as seen in languages like C,
\Cplusplus{}, Java, \Csh{}, Objective-C, JavaScript, Python, Ruby či PHP.

These languages are based on statements, which are sequentially executed and
change the program's state. In contrast, functional languages treat computations
as evaluation of expressions. The fundamental way in which basic building blocks of
functional languages -- functions -- differ from their imperative counterparts
(procedures, methods or C-style "functions") is that their result depends
entirely on their arguments. This makes the functions used in programming much
closer to functions used in mathematics.

Algorithms in a functional language are not a sequence of steps describing
\emph{how} the result should be computed, but they rather specify \emph{what}
the result is.

In this paper, we present a solution to two tasks from the final of The Czech
Programming Contest held in 2010 and 2012, using the functional programming
language Haskell. The goal of the paper is to make the reader familiar with the
high-level concepts used in functional programs and their usage in solving
"real" problems.


\paragraph*{Keywords}
Functional programming; Haskell; turtle graphics; AI.

\end{adjustwidth}
