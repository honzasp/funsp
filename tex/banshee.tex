\chapter{Bílá paní: cesta hradem}
\label{chap:banshee}

\begin{quotation}
Bílá paní s bezhlavým rytířem společně sledovali bezhlavě prchající turisty.
\uv{A nenudíte se sama dole v podhradí,} ptal se rytíř, \uv{když všechny
návštěvníky vyděsíte tak, že nestačí utíkat?} \uv{Ale ne, jedni mi tam nechali
televizor a já na něm sleduji ty, jak se to jmenuje ... telenovely. Jenom teď
musím ten televizor zase najít a v mém věku už není procházení zdí tak
jednoduché jako zamlada.}
\end{quotation}

Soutěžní úloha z roku 2010 \cite{banshee-task} (přetisknuta na straně
\pageref{pdf:banshee}) je variací na klasické téma hledání cesty v bludišti.
Bílá paní se nachází na hradě se zdmi a chodbami a snaží se dojít k televizoru,
aby stihla začátek své oblíbené telenovely. Není ovšem sama -- v hradě se
pohybují zvědaví zvědové, kteří se snaží bílou paní odhalit, proto se jim musí
vyhnout. Bílá paní sice může procházet zdmi, ale snaží se procházení omezit na
minimum, dá přednost delší cestě s menším počtem zdí než kratší cestě s větším
počtem zdí.

\section{Popis úlohy}

Bludiště představuje hrad sestávající z volných políček a zdí, ve kterém se po
předem daných cyklických drahách pohybují zvědové. Je dána počáteční pozice bílé
paní a televizor, který představuje kýžený cíl. Bílá paní může procházet zdmi,
ale nesmí být objevena zvědem.

Cesta s počtem kroků $k_1$, která vede skrz $z_1$ zdí, je \emph{lepší} než cesta
s $k_2$ kroky a $z_2$ zdmi právě tehdy, je-li $z_1 < z_2$ nebo $z_1 = z_2 \land
k_1 < k_2$. Platí-li $z_1 = z_2 \land k_1 = k_2$, považujeme obě cesty za
rovnocenné. Úkolem je najít \emph{nejlepší} cestu z počáteční pozice k
televizoru, aniž by byla objevena zvědem.

Zvěd bílou paní objeví tehdy, když
\begin{inparaenum}[(a)]
\item se oba ve stejném čase nachází na stejném políčku, nebo
\item si v jednom kroku vymění pozice, tj. v čase $t$ stojí zvěd na políčku $A$
  a bílá paní na políčku $B$ a v čase $t+1$ bílá paní na políčku $A$ a
  \emph{stejný} zvěd na políčku $B$.
\end{inparaenum}

Vstupem programu je soubor s bludištěm uloženým v klasickém formátu
programovacích soutěží, tj. na prvním řádku dvě čísla určující rozměry bludiště
a na dalších řádcích po znacích jednotlivá políčka -- `@t{.}' volné, `@t{X}'
zeď, `@t{\&}' počáteční pozice bílé paní, `@t{\#}' televizor. Zvěd je uložen
jako znak `@t{@@}', za kterým následuje jeho trasa jako sada pohybů `@t{v}',
`@t{>}', `@t{\^}' a `@t{<}' (na jih, východ, sever a západ). 

Příklad mapy i s pohyby zvědů je na obrázku \ref{fig:banshee-castle-example}.

\begin{figure}
  \centering
  \begin{subfigure}{0.25\textwidth}
    \begin{bansheecastle}
    5 5
    &X...
    XX.@^>v<.
    .X.X.
    @>><<..X@^^<>vv
    ..X#.
    \end{bansheecastle}
    \caption{Soubor s bludištěm}
    \label{fig:banshee-castle-example-file}
  \end{subfigure}
~
  \begin{subfigure}{0.12\textwidth}
    \begin{bansheecastle}
 .X...
 XX.@.
 .X.X.
 @..X@
 ..X..
    \end{bansheecastle}
    \caption{Čas 0}
    \label{fig:banshee-castle-example-step0}
  \end{subfigure}
  ~
  \begin{subfigure}{0.12\textwidth}
    \begin{bansheecastle}
 .X.@.
 XX...
 .X.X@
 .@.X.
 ..X..
    \end{bansheecastle}
    \caption{Čas 1}
    \label{fig:banshee-castle-example-step1}
  \end{subfigure}
  ~
  \begin{subfigure}{0.12\textwidth}
    \begin{bansheecastle}
 .X..@
 XX..@
 .X.X.
 ..@X.
 ..X..
    \end{bansheecastle}
    \caption{Čas 2}
    \label{fig:banshee-castle-example-step2}
  \end{subfigure}
  ~
  \begin{subfigure}{0.12\textwidth}
    \begin{bansheecastle}
 .X...
 XX.@@
 .X.X.
 .@.X.
 ..X..
    \end{bansheecastle}
    \caption{Čas 3}
    \label{fig:banshee-castle-example-step3}
  \end{subfigure}
  ~
  \begin{subfigure}{0.12\textwidth}
    \begin{bansheecastle}
 .X...
 XX.@@
 .X.X.
 @..X.
 ..X..
    \end{bansheecastle}
    \caption{Čas 4}
    \label{fig:banshee-castle-example-step4}
  \end{subfigure}
  ~
  \begin{subfigure}{0.12\textwidth}
    \begin{bansheecastle}
 .X.@.
 XX...
 .X.X@
 .@.X.
 ..X..
    \end{bansheecastle}
    \caption{Čas 5}
    \label{fig:banshee-castle-example-step5}
  \end{subfigure}
  ~
  \begin{subfigure}{0.12\textwidth}
    \begin{bansheecastle}
 .X..@
 XX...
 .X.X.
 ..@X@
 ..X..
    \end{bansheecastle}
    \caption{Čas 6}
    \label{fig:banshee-castle-example-step6}
  \end{subfigure}
  ~
  \begin{subfigure}{0.12\textwidth}
    \begin{bansheecastle}
 .X...
 XX..@
 .X.X@
 .@.X.
 ..X..
    \end{bansheecastle}
    \caption{Čas 7}
    \label{fig:banshee-castle-example-step7}
  \end{subfigure}
  ~
  \begin{subfigure}{0.12\textwidth}
    \begin{bansheecastle}
 .X...
 XX.@@
 .X.X.
 @..X.
 ..X..
    \end{bansheecastle}
    \caption{Čas 8}
    \label{fig:banshee-castle-example-step8}
  \end{subfigure}
  ~
  \begin{subfigure}{0.12\textwidth}
    \begin{bansheecastle}
 .X.@.
 XX.@.
 .X.X.
 .@.X.
 ..X..
    \end{bansheecastle}
    \caption{Čas 9}
    \label{fig:banshee-castle-example-step9}
  \end{subfigure}
  ~
  \begin{subfigure}{0.12\textwidth}
    \begin{bansheecastle}
 .X..@
 XX..@
 .X.X.
 ..@X.
 ..X..
    \end{bansheecastle}
    \caption{Čas 10}
    \label{fig:banshee-castle-example-step10}
  \end{subfigure}
  ~
  \begin{subfigure}{0.12\textwidth}
    \begin{bansheecastle}
 .X...
 XX..@
 .X.X@
 .@.X.
 ..X..
    \end{bansheecastle}
    \caption{Čas 11}
    \label{fig:banshee-castle-example-step11}
  \end{subfigure}

  \caption{Příklad bludiště (převzatý z ukázkových souborů ze soutěže) s
  rozfázovanými pohyby zvědů. Zvědové mají pohyby s periodou 4 a 6, což znamená,
  že po 12 krocích jsou všichni znovu na stejných pozicích.}
  \label{fig:banshee-castle-example}

\end{figure}


\section{Banshee}

I když bychom mohli do angličtiny přeložit \uv{bílou paní} jako ``white lady'',
budeme o ní v programu hovořit jako o ``banshee'', což je v irské mytologii duch
ženy přicházející z podsvětí a předznamenávající blížící se smrt. Samozřejmě
nejde o ekvivalent klasické české hradní bílé paní, ale pro naše účely je název
@t{Banshee} vhodnější než @t{WhiteLady}.

\section{Analýza}

Stejně jako u Krunimíra i nyní bude vhodné program rozčlenit do modulů:

\begin{description}

  \item @t{Banshee.Castle} definuje datové typy reprezentující hrad, které
    budeme používat v celém programu.

  \item @t{Banshee.CastleParser} poskytuje funkci @t{parseCastle}, která přečte
    popis hradu ze souboru ve výše uvedeném formátu.

  \item @t{Banshee.Navigate} obsahuje jádro programu -- funkci @t{navigate},
    která najde cestu v předaném hradu.

  \item @t{Banshee.Main} exportuje @t{main}, která slouží jako uživatelské
    rozhraní programu.

\end{description}

\input{banshee/Banshee/Castle.lhs}
\input{banshee/Banshee/CastleParser.lhs}
\input{banshee/Banshee/Navigate.lhs}
\input{banshee/Banshee/Main.lhs}

\section{Závěr}

Zde prezentované řešení soutěžní úlohy je téměř kompletní, podobně jako u
Krunimíra jsme ale vynechali část s grafickým uživatelským rozhraním (editorem
hradu). Na soutěži by nejspíše obdrželo přibližně 60 bodů z 90 (zbývajících 30
je za editor hradu) a možná nějaký bonus za interaktivní režim.

Nejdůležitější vlastností u podobných úloh je vždy rychlost. V tomto ohledu je
naše řešení velmi dobré -- všechny testovací soubory ze složek
@t{test/data_ukazkova/} a @t{test/data_testovaci/} se na vývojovém
počítači\footnote{AMD Athlon\texttrademark{} 64 X2 Dual Core, frekvence
procesoru 1000~MHz.} vyřeší za necelé dvě sekundy (z toho asi 0.4~s zabere
nejnáročnější soubor @t{data_testovaci/vstupy/4se_zvedy_skrz_zdi/vstup8.in}),
spotřeba paměti se pohybuje okolo 10~MB.

Program je zároveň poměrně krátký, obsahuje méně než 300 řádků kódu (bez modulu
@t{Banshee.Interactive}), včetně přívětivého uživatelského rozhraní.

\subsection{Zdrojové kódy}

Soubory související s úlohou Bílá paní se nachází ve složce @t{banshee/} v
repozitáři s prací. Zdrojové kódy všech modulů jsou uloženy ve složce
@t{banshee/Banshee/}, testovací soubory ve složce @t{banshee/test/}. Část těchto
souborů pochází ze soutěže a část byla vytvořena v rámci této práce. Konkrétně
ve složce @t{test/extra/huge/} se nachází velmi velké hrady a ve složce
@t{test/extra/cellular/} jsou hrady vytvořené pomocí celulárních automatů.

K automatickému testování korektnosti řešení je možné využít program
@t{runtests.hs}, který čte názvy souborů s hrady a očekávané výsledky ze
souboru @t{test/results.json} a kontroluje, jestli nalezená řešení témto
očekávaným výsledkům odpovídají.
