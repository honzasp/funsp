\begin{figure}
  \centering
  \begin{subfigure}{0.25\textwidth}
    \begin{bansheecastle}
    5 5
    &X...
    XX.@^>v<.
    .X.X.
    @>><<..X@^^<>vv
    ..X#.
    \end{bansheecastle}
    \caption{Soubor s bludištěm}
    \label{fig:banshee-castle-example-file}
  \end{subfigure}
~
  \begin{subfigure}{0.12\textwidth}
    \begin{bansheecastle}
 .X...
 XX.@.
 .X.X.
 @..X@
 ..X..
    \end{bansheecastle}
    \caption{Čas 0}
    \label{fig:banshee-castle-example-step0}
  \end{subfigure}
  ~
  \begin{subfigure}{0.12\textwidth}
    \begin{bansheecastle}
 .X.@.
 XX...
 .X.X@
 .@.X.
 ..X..
    \end{bansheecastle}
    \caption{Čas 1}
    \label{fig:banshee-castle-example-step1}
  \end{subfigure}
  ~
  \begin{subfigure}{0.12\textwidth}
    \begin{bansheecastle}
 .X..@
 XX..@
 .X.X.
 ..@X.
 ..X..
    \end{bansheecastle}
    \caption{Čas 2}
    \label{fig:banshee-castle-example-step2}
  \end{subfigure}
  ~
  \begin{subfigure}{0.12\textwidth}
    \begin{bansheecastle}
 .X...
 XX.@@
 .X.X.
 .@.X.
 ..X..
    \end{bansheecastle}
    \caption{Čas 3}
    \label{fig:banshee-castle-example-step3}
  \end{subfigure}
  ~
  \begin{subfigure}{0.12\textwidth}
    \begin{bansheecastle}
 .X...
 XX.@@
 .X.X.
 @..X.
 ..X..
    \end{bansheecastle}
    \caption{Čas 4}
    \label{fig:banshee-castle-example-step4}
  \end{subfigure}
  ~
  \begin{subfigure}{0.12\textwidth}
    \begin{bansheecastle}
 .X.@.
 XX...
 .X.X@
 .@.X.
 ..X..
    \end{bansheecastle}
    \caption{Čas 5}
    \label{fig:banshee-castle-example-step5}
  \end{subfigure}
  ~
  \begin{subfigure}{0.12\textwidth}
    \begin{bansheecastle}
 .X..@
 XX...
 .X.X.
 ..@X@
 ..X..
    \end{bansheecastle}
    \caption{Čas 6}
    \label{fig:banshee-castle-example-step6}
  \end{subfigure}
  ~
  \begin{subfigure}{0.12\textwidth}
    \begin{bansheecastle}
 .X...
 XX..@
 .X.X@
 .@.X.
 ..X..
    \end{bansheecastle}
    \caption{Čas 7}
    \label{fig:banshee-castle-example-step7}
  \end{subfigure}
  ~
  \begin{subfigure}{0.12\textwidth}
    \begin{bansheecastle}
 .X...
 XX.@@
 .X.X.
 @..X.
 ..X..
    \end{bansheecastle}
    \caption{Čas 8}
    \label{fig:banshee-castle-example-step8}
  \end{subfigure}
  ~
  \begin{subfigure}{0.12\textwidth}
    \begin{bansheecastle}
 .X.@.
 XX.@.
 .X.X.
 .@.X.
 ..X..
    \end{bansheecastle}
    \caption{Čas 9}
    \label{fig:banshee-castle-example-step9}
  \end{subfigure}
  ~
  \begin{subfigure}{0.12\textwidth}
    \begin{bansheecastle}
 .X..@
 XX..@
 .X.X.
 ..@X.
 ..X..
    \end{bansheecastle}
    \caption{Čas 10}
    \label{fig:banshee-castle-example-step10}
  \end{subfigure}
  ~
  \begin{subfigure}{0.12\textwidth}
    \begin{bansheecastle}
 .X...
 XX..@
 .X.X@
 .@.X.
 ..X..
    \end{bansheecastle}
    \caption{Čas 11}
    \label{fig:banshee-castle-example-step11}
  \end{subfigure}

  \caption{Příklad bludiště (převzatý z ukázkových souborů ze soutěže) s
  rozfázovanými pohyby zvědů. Zvědové mají pohyby s periodou 4 a 6, což znamená,
  že po 12 krocích jsou všichni znovu na stejných pozicích.}
  \label{fig:banshee-castle-example}

\end{figure}
