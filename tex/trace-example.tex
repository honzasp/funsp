\begin{figure}

\begin{tikzpicture}[
  image/.style={x=.08mm,y=-.08mm},
  l2/.style={image,line width=.2mm},
  l4/.style={image,line width=.4mm},
  segline/.style={image,-to,shorten >=2mm,draw=black!60,thin,dashed},
  sibling distance=5cm,
  trace/.style={font=\ttfamily},
  segtr/.style={trace},
  split/.style={trace},
  empty/.style={trace},
  seg/.style={font=\ttfamily\footnotesize,node distance=-3mm and -13mm,
    text width=35mm},
  edge from parent/.style={draw,-to},
  level distance=15mm,
]

\node[segtr] (n1) at (-6.5,0) {SegmentTrace}
  child {node[split] {SplitTrace}
    child {node[segtr] (n2) {SegmentTrace}
      child {node[segtr] (n3) {SegmentTrace}
        child { node[segtr] (n4) {SegmentTrace}
          child { node[empty] {EmptyTrace}}
        }
      }
    }
    child {node[segtr] (n5) {SegmentTrace}
      child { node[empty] {EmptyTrace} }
    }
  }
;

\node[seg,below right=of n1] (s1) {Segment (350,500) (350,350) (0,0,0) 2} ;
\node[seg,below right=of n2] (s2) {Segment (200,200) (200,100) (0,0,0) 2} ;
\node[seg,below right=of n3] (s3) {Segment (350,100) (500,100) (0,0,0) 4} ;
\node[seg,below right=of n4] (s4) {Segment (350,350) (600,100) (0,0,0) 4} ;
\node[seg,below right=of n5] (s5) {Segment (500,100) (600,200) (127,127,127) 4} ; 

\draw [image] (0,0) rectangle (701,701) ;
\draw [l2] (350,500) -- (350,350) ;
\draw [l2] (200,200) -- (200,100) ;
\draw [l4] (350,350) -- (600,100) ;
\draw [l4] (350,100) -- (500,100) ;
\draw [l4,color=black!50] (600,200) -- (500,100) ;

\draw [segline] (s1.south) to [bend right=10] (350,425) ;
\draw [segline] (s2.north) to [bend left=20] (200,150) ;
\draw [segline] (s3.south) .. controls +(10cm,0) .. (475,225) ;
\draw [segline] (s4.north) to [bend right=20] (425,100) ;
\draw [segline] (s5.north) to [bend left=5] (550,150) ;

\end{tikzpicture}

% there might be a cleaner solution
\begin{adjustwidth}{2cm}{0cm}
\begin{haskell}[basicstyle=\ttfamily\footnotesize]
SegmentTrace (Segment (350,500) (350,350) (0,0,0) 2)
  (SplitTrace
    (SegmentTrace (Segment (200,200) (200,100) (0,0,0) 2)
      (SegmentTrace (Segment (350,100) (500,100) (0,0,0) 4)
        (SegmentTrace (Segment (500,100) (600,200) (127,127,127) 4)
          EmptyTrace)))
    (SegmentTrace (Segment (350,350) (600,100) (0,0,0) 4)
      EmptyTrace))
\end{haskell}
\end{adjustwidth}

\caption{Grafické znázornění stopy. Obrázek nalevo odpovídá vykreslené stopě,
šipky od jednotlivých segmentů ukazují na příslušné úsečky na obrázku.  Důležitý
je způsob, jakým se čáry překrývají -- segment, jenž je blíž kořeni stromu, je
překryt segmentem nacházejícím se ve stromu \uv{hlouběji}.}
\label{fig:krunimir-trace}

\end{figure}
